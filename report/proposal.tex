\input{config/preamb}
\title{Software 5: "Embedded systems"} % Title
\date{19. December 2019}                                                        % Date
\def\groupnumber{Group SW508e19}                                                  % Group name
\author{\groupnumber}                                                             % Author

\let\theauthor\author
\let\thedate\date

\addbibresource{bibtex/litteratur.bib}


\begin{document}

%\begin{titlingpage}
%  \include{titlepage}
%  \input{Titelblad.tex}
%\end{titlingpage}

%\input{resume.tex}
%\listoffigures
%\newpage
%\listoftables
%\newpage
%\tableofcontents

% vvv Text content goes here vvv

\chapter*{Project Proposal}
In order to express, and expand on, our idea for a project to work on during the 5\textsuperscript{th} semester of Software, the following short proposal/explanation has been written.

\section*{Initial Idea}
Through a couple of discussions and venting of ideas, we ended up in a theme of 'autonomous vehicles' in some form, and narrowed our field down to \textit{self driving cars}.
This field seems to have plenty of interesting problems to work with, both in terms of easy-to-achieve goals for a minimum viable product (MVP), and challenging expansions for showing off our capabilities.

\section*{Relevance}
It is commonly known that cars (and other vehicles) have been a part of society for a long time, which means they are, without a doubt, staying in society for the foreseeable future.
It is also commonly known that it takes training for a human being to be able to steer a cars safely, which means that a certain danger is connected to the use of cars, as well as a mental strain for the driver behind the wheel.
Both of these issues makes it relevant to work on self driving cars that can be deployed, and drive safely enough, without any further training than what was programmed at the factory.

\section*{Possible Problems to Solve}
Through a problem analysis, a list resembling the following list of achievable goals for a car will be presented:
\begin{itemize}
	\item Capable of driving forwards, backwards, and turning. (Not yet MVP)
	\item Capable of driving autonomously within two straight lines drawn on the ground. (MVP)
	\item Capable of driving autonomously on a looping track without cross sections. (MVP+)
	\item Capable of driving autonomously on a looping track with cross sections.
	\item Capable of avoiding collision with stationary obstacles while driving.
	\item Capable of avoiding collision with moving obstacles while driving.
	\item Detect, and stay within, changing speed limits.
\end{itemize}
As stated above, the list is subject to change during the actual problem analysis, but should still contain problems and goals similar to the ones presented.
It is also worth mentioning that the list should be seen more as a backlog than an actual to-do list.
The problem analysis will clearly present which elements from the list we will solve during the semester.

% ^^^ Text content goes here ^^^

\begingroup
	\raggedright
	%\bibliographystyle{bibtex/harvard}
  \printbibliography
\endgroup
\cleardoublepage

\end{document}
