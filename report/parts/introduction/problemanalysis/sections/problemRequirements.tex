\section{Problem definition}\label{sec:problem-statement}
In \autoref{probana:problemsTraffic} it's found that there were problems in traffic.
These problems were, to some degree, caused by humans and their behavior in traffic.
Other problems were related to the infrastructure of the roads.
A level 5 AV could remove the need for humans in traffic, meaning the human influence in traffic could be eliminated.
This could significantly impact the traffic problems related to humans.
Additionally, it could also influence other problems related to traffic congestion.
For an AV system to be successful, several problems need to be solved.
The general problems an AV system faces are described in \autoref{sec:probAnal-tasksAV}, while more specific problems are described in \autoref{sec:implementationIssuesAV} and \autoref{sec:stateOfTheArt}.
Due to the size of the subject, it has been decided to focus on a smaller subset of the problems faced by AVs.


%RESOLVED\todo{Nævn at vi ikke vil solve de mere specifikke problemer ( i state of the art og implementation issues) men en overall løsning af de generelle problemer såsom lane identification etc...}

Therefore the following problem statement has been developed.

\begin{colbox}{D3D3D3}{Problem Statement:}
    {How can we design and implement a prototype of an AV, that can identify and drive within its lane?}
\end{colbox}

\subsection{Problem requirements}\label{ssec:problemRequirements}
%formål med projektet ( dvs. hvad er meningen med projektet, og hvad ønsker vi at ende ud med. Hvad vi prøver at forsøge med projektet og vil ende ud med. )
The intent of this subsection is to form requirements for the project which, when fulfilled, will satisfy the purpose of the project and the problem statement.

The purpose of this project is to design and implement an embedded system.
This is reflected in this project by creating an autonomous driving system which will be embedded in a lego vehicle.

%(Heri skal der være hvad formålet med underafsnittet er.)
In order to determine whether the purpose and problem statement of the project has been reached, these requirements have been found:
\begin{itemize}
    \item[$\square$] A working lego vehicle similar to a car
    \item[$\square$] An autonomous driving system which is capable of driving and following its lane
\end{itemize}

%argumentation for requirements ( this means that we elaborate on the requirements found, (problemstilling vi fandt frem til)
%a working vehicle similar to cars
As the vehicle is a model of an AV, it is important that the it operates in a similar manner to a car.
For the purposes of the vehicle, this includes having four wheels, steering, and control of speed/acceleration.

%The vehicle should be capable of driving and following its lane
The goal of the project is to create a working prototype of an AV system that can drive on its own, which implies a necessity for the vehicle to be capable of following a road.
The road can either be straight or turn to the left or right, but does not include more advanced things such as intersections or roundabouts.
The lane will be represented by two same colored lines with the color being in contrast to the ground.
The colors do not have to be white and black/gray, as they typically are on regular paved roads.

When these requirements have been achieved, the problem statement is considered resolved.
