\subsection{What can be done?}\label{probana:theSolution}
This section examines the findings of \autoref{probana:problemsTraffic}, in an attempt to find a way of reducing the problems in traffic.
This is done to narrow down the problem domain to a more manageable size.

In the previous \autoref{probana:problemsTraffic}, two issues (congestion and accidents) in traffic were described and analyzed.
It was found that these issues can lead to serious problems.
These problems were anything from delays to loss of human life.
While there were various factors involved in causing these problems,
one factor that seemed to have an especially big impact was human error.
It was seen that human errors were very frequent in traffic accidents, and while it did not have the same impact on traffic congestion, it still was a factor.
One way to lower the number of human errors in traffic would be to reduce the influence of people in traffic.
This can be done with self-driving cars or encouraging people to use public transport.

A self-driving car is a car that can control, or partially control itself, independent of human intervention.
A more in-depth description of self-driving cars can be found in \autoref{probana:self-drivingCars}.
By implementing self-driving cars into traffic, it would be possible to lower the amount of human errors, or even entirely remove them, with a fully self-driving car.
Additionally, using fully self-driving cars would enable the previous driver, now passenger, to use the time spent commuting on other activities.
A fully self-driving car could potentially reduce the amount of hours spent in traffic by 250 million, in the 50 most congested cities\cite{lanctot_accelerating_2017}.

If more people start using public transport, it would mean fewer private cars on the road, leading to less traffic.
It would also mean that fewer people have a direct influence on traffic, as they are now passengers on a bus or train.
There is, however, no guarantee that people are willing to use public transport over their own vehicles, as public transport tends to be less flexible, e.g. busses drive on fixed routes between bus stops.
Even with less people in traffic, bus drivers are still capable of making mistakes, and any mistakes leading to accidents could endanger the bus passengers.

It is already possible to find a degree of self-driving technology in some cars today \cite{tesla_technology}.
Despite the possibility of lowering the number of human errors, there is currently no full fledged self-driving car available commercially.
As such the next section will further investigate what a self-driving car is.
