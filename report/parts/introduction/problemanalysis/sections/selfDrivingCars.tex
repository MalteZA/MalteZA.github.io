\section{Self-driving cars}\label{probana:self-drivingCars}
There are different factors that can have an impact on the way traffic flows and in some cases whether people get hurt or not.
One factor that is seen in all vehicles on the road today is the human.
This factor cannot be precisely determined, because not all humans react with the same speed or in the same way.
Other things, like alcohol or lack of sleep, can in most cases have a negative effect on this factor.
\cite{vi.chilukuri.ctrdot.gov_usdot_2017}

The human factor can therefore make traffic and vehicles more or less unpredictable.
If an intelligent system should control and maybe predict traffic flow, to avoid congestion and accidents, the vehicles on the road must be as close to completely predictable as possible.
A possible solution to this, is to remove the human factor from the equation, by making the car drive itself.

\subsection{The six levels of self-driving cars}
There are 6 different levels of self-driving technology according to the US National Highway Traffic Safety Administration \cite{matthew.lynberg.ctrdot.gov_automated_2017}.
These different levels are as follows:

A \textbf{level 0} vehicle is not able to anything by itself.
The driver is in charge of everything that is happening.

At \textbf{level 1}, the vehicle is able to assist the driver.
An example of a feature that could be a part of a level 1 vehicle is adaptive cruise control.

When multiple automated features are combined it is a \textbf{level 2} vehicle.
At this level the vehicle could, as an example, steer direction and control acceleration at the same time.
The driver must be able to take control of the vehicle at any time, though.

At \textbf{level 3}, the vehicle is able to monitor the environment around the vehicle.
If the vehicle get into a situation that might be confusing, the vehicle will return the steering to the driver.

\textbf{Level 4} describes an almost self-driving car.
It is able to read the environment and take action upon these inputs in most conditions.
There will be situations where the driver needs to take control of the vehicle.

For a vehicle to be 100 percent self-driving it needs to be at \textbf{level 5}.
At this level, the vehicle is able to be in control and handle every situation, no matter the conditions.
The driver is still able to take control of the vehicle, but there will never be a situation where this is necessary, this means that the human factor can be eliminated.

Most production vehicles do not exceed level 2, with a few exception like the Audi R8 which is a level 3 vehicle.
Even Tesla's Autopilot is categorized as a level 2.
This does not mean that no high level Autonomous Vehicle (AV) exists, however.
Companies like Google have developed level 4 vehicles and several concept cars are at level 5.
The reasons for the absence of high level production AVs will be discussed in \autoref{sec:implementationIssuesAV}.
\cite{6_levels_with_examples}
